% !TEX encoding = UTF-8 Unicode

% Beispiel für ein LaTeX-Dokument im Format "seminarvorlage"
\documentclass[ngerman]{seminarvorlage}
% ngerman = Deutsch in neuer Rechtschreibung, alternativ english
\usepackage{babel}[german] % automatische Sprachunterstützung

\usepackage[utf8]{inputenc} % Kodierung der Non-ASCII-Zeichen
\usepackage[T1]{fontenc} % Moderne Fonts, Trennung von Wörtern mit Umlauten
\usepackage{cleveref} % für bequeme Referenzen, siehe \cref unten

\usepackage{embrac}% upright brackets in emphasised text, () and [], empfohlen

%\usepackage{biblatex}% besser als bibtex, aber dann biber statt bibtex benutzen
\begin{document}

% Unbedingt angeben: Titel, Autoren
% Freiwillig: Adresse, E-Mail
\title{Vergleich binärer und nicht-binärer Serialisierungsformate}
\numberofauthors{2}
\author{
  \alignauthor Matthias Raba \\
    \email{i22029@hb.dhbw-stuttgart.de}
  \alignauthor Fabian Noll\\
    \email{i22024@hb.dhbw-stuttgart.de}
}

\maketitle% Titelangaben produzieren, aber kein Inhaltsverzeichnis, \tableofcontents funktioniert nicht!
\newpage

\abstract{In dieser Arbeit vergleichen wir binäre und nicht-binäre Serialisierungsformate mit dem Ziel, ihre Eignung für unterschiedliche Anwendungsbereiche zu evaluieren. Im Fokus stehen die Formate JSON, XML, ProtoBuf und MessagePack, die wir hinsichtlich ihrer Geschwindigkeit, Speicherbedarf und Wartbarkeit/Nutzbarkeit untersuchen. Durch eine Reihe von Tests und Analysen bewerten wir die Leistung der Formate in verschiedenen Szenarien und beleuchten deren Stärken und Schwächen. Darüber hinaus erörtern wir die jeweiligen Vor- und Nachteile in spezifischen Anwendungsgebieten, um eine fundierte Entscheidungsgrundlage für die Auswahl des passenden Serialisierungsformats in der Praxis zu bieten.}

\keywords{Serialisierungsformate, JSON, XML, ProtoBuf, 
MessagePack, Binäre Serialisierung, Nicht-binäre Serialisierung,
Datenformatvergleich, Anwendungsgebiete}

% Section-Überschriften werden automatisch in GROSSBUCHSTABEN gesetzt
\section{Einleitung}

% Jede Section am besten mit einem Kommentar hier im Quelltext markieren
\section{Grundlagen}

% Kurze Definition von Serialisierung und Deserialisierung
\subsection{Serialisierung und Deserialisierung}
Serialisierung bezeichnet den Prozess der Umwandlung komplexer Datenstrukturen oder Objekte in ein lineares Datenformat, das für die Speicherung oder Übertragung in verschiedenen Medien geeignet ist, wie beispielsweise Dateien oder Netzwerkprotokolle. Deserialisierung hingegen beschreibt den Vorgang, bei dem das lineare Datenformat analysiert und in die ursprüngliche Datenstruktur oder das objektorientierte Repräsentationsmodell zurückgeführt wird. Diese Prozesse sind essenziell für die Persistenz von Daten und die Kommunikation zwischen verteilten Systemen.

% Kurze Gegenüberstellung von binären und nicht binären Formaten, Vor- und Nachteile
\subsection{Binäre vs. Nicht-Binäre Formate}
Binäre und nichtbinäre Serialisierungsformate unterscheiden sich grundlegend in der Art und Weise, wie sie Daten speichern und übertragen. Binäre Formate wie Protobuf oder MessagePack codieren Daten in einem kompakten, maschinenlesbaren Format, was ihnen eine hohe Effizienz sowohl hinsichtlich der Speichernutzung als auch der Verarbeitungsgeschwindigkeit verleiht. Diese Formate sind ideal für Systeme, in denen Performance eine zentrale Rolle spielt, wie bei der Kommunikation zwischen Microservices oder in Anwendungen mit begrenzter Bandbreite. Ein Nachteil dieser Formate ist jedoch, dass sie nicht von Menschen direkt gelesen werden können. Dies bedeutet, dass spezielle Tools oder Software nötig sind, um den Inhalt zu interpretieren, was die Fehlersuche und das Debuggen erschweren kann.

Nichtbinäre Formate wie XML oder JSON speichern Daten in einem textbasierten, menschenlesbaren Format. Ein klarer Vorteil ist die einfache Lesbarkeit und Bearbeitbarkeit, da Entwickler und Nutzer ohne spezielle Software die Inhalte direkt inspizieren und ändern können. Diese Eigenschaft macht sie besonders geeignet für Konfigurationsdateien, Datenformate, die von Menschen überprüft werden müssen, oder APIs, bei denen Klarheit und Transparenz im Vordergrund stehen. Der Nachteil liegt in der ineffizienteren Speichernutzung und der geringeren Performanz, da diese Formate mehr Speicherplatz benötigen und der Parsing-Prozess langsamer ist als bei binären Formaten. XML zum Beispiel erzeugt durch seine umfangreichen Tags oft eine größere Datenmenge, während JSON kompakter ist, aber dennoch die Nachteile von Textformaten teilt.

In Bezug auf den Einsatzbereich sind binäre Formate aufgrund ihrer Effizienz und hohen Geschwindigkeit in Bereichen wie Embedded Systems, Kommunikationsprotokollen und Cloud-basierten Diensten weit verbreitet. Nichtbinäre Formate hingegen finden häufig in Bereichen Anwendung, in denen eine einfache Handhabung und schnelle Lesbarkeit durch Menschen wichtiger sind als die absolute Performance, wie z.B. bei Datenübertragungen im Web oder bei einfachen Konfigurationsdateien. Während binäre Formate eine bessere Performance und Speicheroptimierung bieten, sind textbasierte Formate flexibler und einfacher in der Nutzung und im Austausch.

\subsection{Serialisierung vs. Komprimierung}



\subsection{Theoretisch optimale Serialisierung}

Jedes Serialisierungsformat hat ein unteres Limit bezüglich des benötigten Speicherplatzes eines serialisierten Datensatzes. Für diese Betrachtung werden folgende Datentypen angenommen: 

\begin{table}[h]
  \centering
  \begin{tabular}{|l|l|l|} 
  \hline
  Datentyp  & Wertebereich               & Arbeitsspeicher                    \\ 
  \hline
  i32         & $-\left(2^{31}-1\right)$ bis $2^{31}$  & 32 Bits              \\
  f32         & $\pm 10^{38}$              & 32 Bits                          \\
  boolean     & {[}true, false]            & 8 Bits                           \\
  $string_n$  & $[0, 1114112]^n$           & $8 \cdot n \leq 32 \cdot n$ Bits \\
  \hline
  \end{tabular}
\end{table}

Die aufgeführten Größen sind in vielen gängigen Programmiersprachen vertreten (C, C++, Java, Rust). Bei Zeichenketten wird von einer UTF-8 kompatiblen Implementation ausgegangen. Moderne Programmiersprachen unterstützen die effizientere Darstellung von mehreren Booleans in einem Byte. Da dies aber nicht in jeder Sprache der Fall ist, wird diese Optimierung nicht berücksichtigt.

Unabhängig vom Serialisierungsformat muss verarbeitende Software wissen, welche Struktur die Daten haben. Diese Struktur kann entweder durch die Syntax des Serialisierungsformats vorgegeben oder durch andere Mechanismen definiert werden. JSON und Protobuf sind jeweils Beispiele für beide Mechanismen. Da JSON ein text-basiertes, nichtbinäres Format ist, ist die Typzugehörigkeit aus dem serialisierten Datum ersichtlich. Die Definition der Syntax von JSON garantiert, dass die Wertebereiche der Datentypen sich nicht überschneiden. Protobuf hat diese Metainformation grundsätzlich nicht, da es auf ``rohen'' binären Datensätzen arbeitet. Die Grenzen sowie die Datentypen der einzelnen Serialisierten Feldern können nicht aus dem Datensatz gewonnen werden. Daher benötigt Protobuf eine externe Datei, die die Struktur der Datensätze beschreibt. Da diese \texttt{.proto} Dateien vergleichsweise klein sind (wenige kB), werden sie bei dieser Betrachtung vernachlässigt. Wird keine solche externe Strukturdefinition verwendet, ist es notwendig, durch festgelegte Zeichenketten, auch ``Delimiter'' genannt, die Grenzen der Datenfelder kennzuzeichnen.

\subsubsection{Binäre Formate}

In binären Formaten werden die primitiven Datentypen effizient abgebildet, da die serialisierte bereits der deserialisierten Form entspricht. Angenommen, die Struktur wird durch eine externe Definition gegeben, entsteht kein Overhead im Vergleich zur nativen Repräsentation im Arbeitsspeicher. Allerdings kann es notwendig sein, die Endianness bei der Serialisierung zu verändern, was keinen Einfluss auf die Größe des Datums hat.

\subsubsection{Nichtbinäre Formate}

Bei nichtbinären Formaten muss eine Transformation zwischen der binären und der textuellen Form der Daten vorgenommen werden. Für diese Transformation gilt:

$$ M_{\text{textuell}} \geq M_{\text{binär}} $$

Angenommen, das Serialisierungsziel unterstützt UTF-8, benötigen Strings keine solche Transformation, da sie direkt übernommen werden können. Viele Serialisierungsformate benötigen allerdings Delimiter, um den Anfang und das Ende des Strings anzuzeigen. Zusätzlich kann es notwendig sein, im Format als Steuerzeichen verwendete Zeichen zu modifizieren, damit sie von den Steuerzeichen unterscheidbar bleiben. Viele Formate verwenden dazu z.B das Präfix `\\' oder `/'.

Die minimale Größe eines einzelnen Zeichens sind 8 Bits. Daraus folgt, dass auch Booleans mindestens mit einem Zeichen repräsentiert werden. Daher gilt auch für Booleans $ M_{\text{textuell}} \geq M_{\text{binär}} $. Viele Formate verwenden die Literale ``true'' und ``false'', um die Lesbarkeit zu erhöhen. Angenommen, beide Werte treten gleich häufig auf, wird die Effizienz damit um den Faktor $\frac{4 + 5}{2} = 4.5$ gesenkt.

Für die Serialisierung von Integern wird ebenfalls von einer Gleichverteilung im Wertebereich angenommen. Zur Berechnung der durchschnittlichen Länge eines solchen Integers muss zunächst die durchschnittliche Anzahl der Dezimalstellen erfasst werden. Die Anzahl der Ganzzahlen mit n Stellen ist durch $10^n$ gegeben. Allerdings werden hierbei führende Nullen nicht berücksichtigt.Folgende Tabelle zeigt die tatsächliche Verteilung der positiven Ganzzahlen:

\begin{table}[h]
  \centering
  \begin{tabular}{|l|l|l|l|} 
  \hline
  \textbf{Untere Grenze} & \textbf{Obere Grenze} & \textbf{Anzahl} & \textbf{Dezimalstellen}  \\ 
  \hline
  0                      & 9                     & 10              & 1                        \\
  10                     & 99                    & 90              & 2                        \\
  100                    & 999                   & 900             & 3                        \\
  1000                   & 9999                  & 9000            & 4                        \\
  10000                  & 99999                 & 90000           & 5                        \\
  100000                 & 999999                & 900000          & 6                        \\
  1000000                & 9999999               & 9000000         & 7                        \\
  10000000               & 99999999              & 90000000        & 8                        \\
  100000000              & 999999999             & 900000000       & 9                        \\
  1000000000             & 2147483648            & 1147483649      & 10                       \\
  \hline
  \end{tabular}
\end{table}

Das arithmetische Mittel der Länge wird wie folgt berechnet:

$$\frac{\sum_{i=1}^{10} Anzahl(i) * Dezimalstellen(i)}{2^{31}} \approx 9.4826$$

Da die Zahlen von 0-9 in UTF-8 mit nur einem Zeichen und damit nur einem Byte repräsentiert werden, sind die durchschnittliche Anzahl an Dezimalstellen und der durchschnittliche Speicherverbrauch der gleichen Zahl identisch. ALlerdings führt die Tabelle nur die positiven Zahlen auf. Negative Zahlen werden üblicherweise über das vorangestellte Symbol `-' deklariert. Da der Zahlenraum der positiven und negativen Zahlen nahezu gleich groß sind, ergibt sich daraus eine durchschnittliche Erhöhung der Länge um 0.5 Zeichen. Somit ist die durchschnittliche Länge eines textuell serialisierten 32-Bit Integers 9.9826 Byte. Da ein nativer Integer in nur 4 Bytes repräsentiert wird, ist die textuelle Repräsentation im Durchschnitt um 149.56\% größer. Somit gilt auch für Integer $ M_{\text{textuell}} \geq M_{\text{binär}} $.

Ähnliche Berechnungen sind auch für Fließkommazahlen möglich, allerdings ist die Repräsentation, im Gegensatz zu den Ganzzahlen, implementationsabhängig. Dabei gibt es verschiedene Merkmale, die festlegen, ab wann die wissenschaftliche Notation verwendet wird. Allerdings lässt bereits der enorme abbildbare Zahlenraum von $\pm10^{38}$ darauf schließen, dass die benötigte Anzahl an Dezimalstellen die der Ganzzahlen weit übersteigt. Zusätzliche Zeichen und Zeichenketten, wie etwa der Dezimalpunkt oder der Exponent in der wissenschaftlichen Notation erhöhen die Anzahl der benötigten Zeichen zusätzlich.

\newpage

\section{State of the Art Vergleiche}

\section{Kriterien}

Zum Vergleich von Serialisierungsverfahren ist die Auswahl geeigneter Metriken entscheidend, um fundierte und vergleichbare Ergebnisse zu erzielen. Die folgenden Metriken wurden festgesetzt und mit Fachliteratur abgeglichen und bieten eine umfassende Grundlage für den Vergleich:

\textbf{Serialisierungszeit}\newline
Diese Metrik misst die Zeit, die benötigt wird, um Datenstrukturen in ein spezifisches Format zu konvertieren. Eine geringere Serialisierungszeit kann die Effizienz von Anwendungen erhöhen, insbesondere in zeitkritischen Systemen. Die Messung erfolgt durch Zeitstempel vor und nach dem Serialisierungsprozess, wobei die Differenz die benötigte Zeit ergibt.

\textbf{Deserialisierungszeit}\newline
Sie erfasst die Dauer, die erforderlich ist, um serialisierte Daten zurück in ihre ursprüngliche Struktur zu überführen. Eine schnelle Deserialisierung ist essenziell für Anwendungen, die zügig auf gespeicherte oder übertragene Daten zugreifen müssen. Die Messmethodik entspricht derjenigen der Serialisierungszeit, mit Fokus auf den Deserialisierungsprozess.

\textbf{Kompressionsrate}\newline
Sie gibt das Verhältnis der Größe der originalen Daten zur Größe der serialisierten Daten an. Eine höhere Kompressionsrate reduziert den Speicherbedarf und die Bandbreitennutzung, was besonders in ressourcenbeschränkten Umgebungen von Vorteil ist. Die Berechnung erfolgt durch Division der Größe der ursprünglichen Daten durch die Größe der serialisierten Daten.

\textbf{Implementierungsgüte}\newline
Hierbei wird untersucht, wie die Performance der Serialisierungsverfahren in verschiedenen Programmiersprachen variiert. Unterschiede in der Implementierung können signifikante Auswirkungen auf die Effizienz haben, weshalb Tests in Sprachen wie Rust und Python aufschlussreich sind. Die Messung erfolgt durch Vergleich der gemessenen Metriken über verschiedene Implementierungen hinweg.

\textbf{Nutzbarkeit/Wartbarkeit}\newline
Diese qualitative Metrik bewertet, wie einfach ein Serialisierungsverfahren zu implementieren, zu nutzen und zu warten ist. Aspekte wie Dokumentation, Community-Unterstützung und die Komplexität der Integration spielen hierbei eine Rolle. Die Bewertung erfolgt durch Experteneinschätzungen und Benutzerfeedback.

\textbf{Ressourcenverbrauch}\newline
Diese Metrik misst die Nutzung von Systemressourcen wie CPU, Speicher und, falls relevant, auch Energie während der Serialisierungs- und Deserialisierungsprozesse. Ein effizienter Ressourcenverbrauch ist entscheidend, insbesondere in Umgebungen mit begrenzten Systemressourcen, wie eingebetteten Systemen oder IoT-Geräten. Diese Metrik wird einfacher bewertet in Bezug auf Probleme oder limitationen die sich während der Testreihen kristallisiert haben.

\section{Die Serialisierungsformate}

Im folgenden Abschnitt wird die Auswahl der zu vergleichenden Formate erläutert als auch die ausgewählten Formate in einer kurzen Zusammenfassung vorgestellt.

\subsection{Begründung der Formatwahl}

Die Auswahl der Formate JSON, XML, Protocol Buffers und MessagePack erfolgte aufgrund ihrer unterschiedlichen Stärken, die jeweils auf spezifische Anwendungsfälle und Anforderungen abzielen.

\textbf{JSON} wurde wegen seiner Einfachheit, Lesbarkeit und breiten Unterstützung in Web-Technologien gewählt. Es ist ideal für Situationen, in denen menschliche Lesbarkeit und einfache Integration in verschiedene Programmiersprachen wichtig sind.

\textbf{XML} wurde aufgrund seiner Flexibilität und umfassenden Unterstützung für strukturierte Daten gewählt. Es ist besonders geeignet für Anwendungen, die eine strenge Validierung und Unterstützung für umfangreiche Metadaten erfordern, wie in vielen Legacy-Systemen oder dokumentenzentrierten Umgebungen.

\textbf{Protocol Buffers} wurden aufgrund ihrer hohen Effizienz und der Fähigkeit, komplexe Datenstrukturen zu serialisieren, ausgewählt. Sie sind ideal für leistungsorientierte Anwendungen, in denen Geschwindigkeit und geringer Speicherbedarf entscheidend sind, wie bei der Kommunikation zwischen Microservices oder in mobilen Anwendungen.

\textbf{MessagePack} ergänzt die Auswahl durch seine Effizienz in der Datenspeicherung und -übertragung, was es zu einer ausgezeichneten Wahl für ressourcenbeschränkte Umgebungen macht. Diese Diversität an Formaten ermöglicht es, verschiedene Szenarien optimal abzudecken, von einfacher Web-Kommunikation bis hin zu hochperformanten, ressourcenoptimierten Anwendungen.

\subsection{Vorstellung der Formate}

\textbf{JSON (JavaScript Object Notation)}
JSON ist ein textbasiertes Serialisierungsformat, das durch seine Einfachheit und Struktur überzeugt. Es speichert Daten in Form von Schlüssel-Wert-Paaren, Arrays und verschachtelten Objekten, die leicht von Maschinen verarbeitet und von Menschen gelesen werden können. JSON hat sich besonders in der Webentwicklung etabliert, da es direkt von JavaScript unterstützt wird, was die Interoperabilität zwischen Frontend und Backend erleichtert. Viele Web-APIs nutzen JSON, um Daten zwischen Client und Server auszutauschen. Die breite Unterstützung in verschiedenen Programmiersprachen macht JSON zu einem universellen Standard für viele Anwendungsfälle. JSON ist nicht auf Webanwendungen beschränkt, sondern findet auch in Konfigurationsdateien, Datenbanken wie MongoDB und bei der Interprozesskommunikation Anwendung. Allerdings hat JSON auch Einschränkungen, darunter ein höherer Speicherbedarf und eine geringere Verarbeitungsgeschwindigkeit im Vergleich zu binären Formaten. Ein weiteres potenzielles Problem ist das Fehlen einer strikten Typisierung, was in größeren Projekten zu Fehleranfälligkeit führen kann. Dennoch bleibt JSON aufgrund seiner Balance zwischen Lesbarkeit und Funktionalität ein unverzichtbares Werkzeug in der modernen Softwareentwicklung.

\textbf{MessagePack}
MessagePack ist ein binäres Serialisierungsformat, das entwickelt wurde, um die Effizienz von Datenübertragungen zu maximieren. Im Gegensatz zu textbasierten Formaten wie JSON oder XML speichert MessagePack Daten in einem kompakten binären Format, was die Dateigröße erheblich reduziert. Dies führt zu schnelleren Übertragungszeiten und geringerer Speichernutzung, was besonders in ressourcenbeschränkten Umgebungen wie mobilen oder eingebetteten Systemen von Vorteil ist. MessagePack unterstützt eine Vielzahl von Datentypen, darunter Strings, Zahlen, Arrays, Maps und sogar Binärdaten, was es vielseitig einsetzbar macht. Ein bedeutendes Merkmal ist die Fähigkeit, die Typinformationen effizient zu kodieren, was eine schnellere Serialisierung und Deserialisierung ermöglicht. Dies ist besonders nützlich in Hochleistungsanwendungen wie Spieleservern, Echtzeitsystemen und IoT-Geräten. Trotz seiner Vorteile in Bezug auf Effizienz hat MessagePack eine eingeschränkte Lesbarkeit für Menschen, was die Fehlerbehebung und Wartung erschweren kann, wenn keine geeigneten Werkzeuge zur Verfügung stehen. Außerdem kann die Komplexität der Implementierung in bestimmten Sprachen höher sein, da spezielle Bibliotheken erforderlich sind, um die binären Daten korrekt zu verarbeiten. Dennoch ist MessagePack eine ausgezeichnete Wahl, wenn die Prioritäten auf Geschwindigkeit, Speicherplatz und Leistung liegen, und es wird zunehmend in Bereichen eingesetzt, in denen JSON aufgrund seiner Textbasiertheit ineffizient wäre.

\textbf{XML (Extensible Markup Language)}
XML ist ein textbasiertes Format, das entwickelt wurde, um Daten in einer strukturierten, hierarchischen Form darzustellen. Es ist sowohl menschen- als auch maschinenlesbar und zeichnet sich durch seine Flexibilität aus, da es keine festen Datentypen vorgibt. XML nutzt Tags, um Daten zu kennzeichnen, was es sehr beschreibend macht, jedoch auch zu einer relativ großen Dateigröße führen kann. Es ist in vielen Bereichen etabliert, vor allem dort, wo Lesbarkeit und Austauschbarkeit wichtig sind, aber es ist im Vergleich zu neueren Formaten weniger effizient in Bezug auf Speicherplatz und Geschwindigkeit.

\textbf{Protobuf (Protocol Buffers)}
Protobuf ist ein binäres Serialisierungsformat, das von Google entwickelt wurde, um effiziente und schnelle Datenübertragung zu ermöglichen. Es zeichnet sich durch seine geringe Größe und hohe Geschwindigkeit bei der Serialisierung und Deserialisierung aus. Im Gegensatz zu XML ist Protobuf nicht direkt menschenlesbar, benötigt aber eine vordefinierte Schema-Datei (".proto"), um Daten zu interpretieren. Dieses Format ist ideal für Systeme, bei denen Leistung und Platzbedarf entscheidend sind, z.B. in Microservices und Netzwerkprotokollen.

\section{Vergleichsmethodik}

\subsection{Genereller Ansatz}

\subsection{Python}

Die Tests wurden in einem strukturierten und standardisierten Ablauf durchgeführt, wobei spezifische Anpassungen vorgenommen wurden, um die Besonderheiten der getesteten Serialisierungsverfahren zu berücksichtigen. Jeder der generierten Datensätze wurde zunächst geladen und seine Größe im Speicher erfasst, um später als Referenz für die Berechnung der Kompressionsrate zu dienen. Anschließend wurden die Serialisierungs- und Deserialisierungszeiten sowie die resultierende Größe der serialisierten Daten für jede Technologie (JSON, XML, MessagePack und Protocol Buffers) gemessen.

Für XML wurde aufgrund des hohen Speicherverbrauchs bei der Verarbeitung größerer Datensätze ein optimiertes Streamingverfahren eingesetzt. Bei diesem Ansatz wird der Datensatz nicht vollständig im Speicher gehalten, sondern schrittweise verarbeitet. Die XML-Daten werden dabei durch eine rekursive Generierung einzelner XML-Elemente Stück für Stück erzeugt und direkt in eine Datei geschrieben. Dieser Mechanismus nutzt einen sogenannten Generator, der jeweils kleine Teile des XML-Dokuments erzeugt, diese sofort ausgibt und sie anschließend aus dem Speicher entfernt. Dadurch wird der Speicherbedarf unabhängig von der Größe des Datensatzes auf ein Minimum reduziert. Dieses Verfahren ermöglicht es, auch extrem große Datensätze zu serialisieren, ohne den verfügbaren Arbeitsspeicher zu überlasten. Es bildet gleichzeitig typische Anwendungsfälle in realen datenintensiven Szenarien ab, bei denen Ressourcen effizient genutzt werden müssen.

Protocol Buffers (Protobuf) erhielt ebenfalls spezifische Implementierungen für jede Datensatzart. Diese Maßnahme war notwendig, da eine generische Behandlung in Protobuf zu schlechten Ergebnissen führt, die die tatsächlichen Stärken des Formats nicht widerspiegeln. Mit maßgeschneiderten Protobuf-Definitionen und entsprechenden Serialisierungs- und Deserialisierungsfunktionen konnte eine realistischere Bewertung vorgenommen werden, die den praxisnahen Einsatz des Formats besser abbildet.

Zusätzlich wurde die Hardwareumgebung detailliert protokolliert, um die Ergebnisse im Kontext der zugrunde liegenden Systemressourcen bewerten zu können. Erfasst wurden dabei unter anderem CPU-Modell, Anzahl der logischen Kerne, verfügbare Arbeitsspeicherkapazität und Betriebssystem. Diese Informationen sind wichtig, da sie maßgeblich die Leistung der Serialisierungsverfahren beeinflussen können. Durch die Dokumentation der Hardwarebedingungen wird die Reproduzierbarkeit der Ergebnisse gewährleistet und es können eventuelle Abweichungen bei Tests auf anderen Systemen besser nachvollzogen werden.

Die Tests wurden für jede Kombination aus Datensatz, Technologie und Implementierung über mehrere Wiederholungen durchgeführt, um statistische Schwankungen zu minimieren. Die Ergebnisse, einschließlich aller gemessenen Metriken, wurden in einer strukturierten JSON-Ausgabe gespeichert.

\subsubsection{Probleme und Limitationen}

Ein wesentliches Problem bei den Messungen war die ineffiziente Datenhaltung in Python. Diese führte dazu, dass ein 680 MB großer JSON-Datensatz im Speicher etwa 2 GB beanspruchte, was die Effizienz der Tests erheblich beeinträchtigte.

Zudem stellten die Implementierungen von Protobuf und XML besondere Herausforderungen dar. Insbesondere XML war aufgrund seines hohen Speicherverbrauchs problematisch. Für einige Datensätze, wie den oben erwähnten JSON-Datensatz, benötigte die XML-Serialisierung teilweise über 30 GB Arbeitsspeicher. Diese Anforderungen führten dazu, dass XML für problematische Datensätze nicht getestet wurde, da die Tests oft unvorhersehbar fehlschlugen. Selbst wenn die Serialisierung funktionierte, betrugen die Laufzeiten mit der Streaming-Variante für große Datensätze häufig mehr als 10 Minuten, was die Tests unpraktikabel machte.

Protobuf brachte ebenfalls spezifische Schwierigkeiten mit sich. Die ineffiziente Datenhaltung in Python, kombiniert mit dessen vergleichsweise langsamer Programmausführungszeit, machte es notwendig, für jeden Datensatz spezifische Protobuf-Implementierungen zu entwickeln, um eine faire Vergleichsbasis zu schaffen. Die generische Implementierung verursachte aufgrund des hohen Sprach-Overheads erheblichen Ressourcenverbrauch und lange Ausführungszeiten, sodass Protobuf in diesen Fällen sogar von XML in Effizienz und Performance übertroffen wurde.

Ein weiteres Problem betrifft die Limitierungen von Python selbst. Bei JSON-Strukturen mit einer Tiefe von mehr als 100 Ebenen führten die zahlreichen Rekursionsaufrufe dazu, dass Python die Verarbeitung mit einem Fehler abbrach. Dieses Problem ließ sich nicht durch eine Erhöhung der maximalen Rekursionstiefe beheben. Ähnliche Probleme traten auch bei Protobuf auf, das mit Datenstrukturen, die mehr als 50 Ebenen tief waren, nicht korrekt umgehen konnte. In diesen Fällen war die Serialisierung fehlerhaft, und die Deserialisierung konnte nicht erfolgreich abgeschlossen werden.

Einzig und allein MessagePack war ein Protokoll das keine Limitationen oder Probleme aufwies.

Diese Einschränkungen zeigen die praktischen Herausforderungen bei der Verwendung von Python und bestimmten Serialisierungsformaten in ressourcenintensiven Szenarien auf. Sie unterstreichen die Bedeutung einer angepassten Implementierung und realistischen Bewertung der Verfahren unter Berücksichtigung ihrer spezifischen Eigenheiten.



\section{Ergebnisauswertung}

\section{Fazit und Ausblick}

In Zukunft wäre es möglich diese Metriken zu erweitern sowie weitere Formate hinzuzufügen als auch die selbe Umgebung in mehreren Programmiersprachen zu testen um etwaige Implementierungsunterschiede durch eine größere Menge an Daten auszumerzen.

% Bibliographie entweder direkt hier eingeben (nur im Notfall)...
%\begin{thebibliography}{9}
%\bibitem{ACM2019}
%ACM.
%\newblock How to classify works using ACM's computing classification system.
%\newblock \url{http://www.acm.org/class/how_to_use.html}.
%
%\bibitem{Ivory2001}
%M.~Y. Ivory and M.~A. Hearst.
%\newblock The state of the art in automating usability evaluation of user
%  interfaces.
%\newblock {\em ACM Comput. Surv.}, 33(4):470--516, 2001.
%
%\end{thebibliography}

% ... oder die Bibliographie mit Hilfe von BibTeX generieren,
% dies ist auf jeden Fall die bessere Lösung und sollte nach
% Möglichkeit immer verwendet werden:
\bibliographystyle{abbrv}
\bibliography{literatur} % Daten aus der Datei literatur.bib verwenden.

\end{document}
